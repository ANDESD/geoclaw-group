
\vskip 5pt
\section{Introduction}

\subsection{Geographic area}  The west coast, with a focus on Washington State,
including Puget Sound, the San Juan Island group, and other islands in the
Strait of Juan de Fuca.


\ignore{
(Describe your NTHMP funded effort regarding inundation mapping, i.e., which
area this covers, which types of tsunami sources are affecting your area and
hence which benchmark problems in the list you will be addressing and why.)
}

\subsection{NTHMP support}
This group has no current or past support from NTHMP.

NTHMP approval of this model will allow us to seek funding in the
future to perform tsunami hazard and risk assessment modeling, such as
from NTHMP partner FEMA’s Risk Mapping,
Assessment, and Planning (Risk MAP) Program.

Funding will also be sought to provide Washington State with tsunami
modeling and mapping in support of tsunami hazard assessment and emergency
management planning and education.  

Tsunami sources in these geographic
areas include earthquakes and landslides, and we will therefore address
benchmark problems that deal with these sources.

\section{Model description}
\ignore{
(Describe and give relevant references to the model(s) you are using for
your tsunami source, propagation, and coastal impact/inundation simulations.
As a minimum, provide general information on the type of models you are
using, including related restriction/assumptions. Refer to (preferably peer
reviewed) papers where your model equations and other relevant validation
and/or case studies for tsunami or otherwise have been published.)
}

\subsection{Model equations}

For all benchmark problems we used the two-dimensional nonlinear
shallow water equations
\begin{equation}\label{2dSWE}
\begin{split}
  h_t &+ (hu)_x + (hv)_y = 0,\\
  (hu)_t &+ (hu^2 + \half gh^2)_x + (huv)_y = -ghB_x - Du,\\
  (hv)_t &+  (huv)_x+ (hv^2 + \half gh^2)_y = -ghB_y - Dv,
\end{split}
\end{equation}
where $u(x,y,t)$ and $v(x,y,t)$ are the depth-averaged velocities in
the two
horizontal directions, $B(x,y,t)$ is the topography or bathymetry,
and $D=D(h,u,v)$ is the drag coefficient.

Most of the benchmark problems use no bottom friction, so $D=0$.  
When used, it has the
form
\eql{drag}
D = \frac{gM^2\sqrt{(u^2+v^2)}}{h^{5/3}}
\end{equation}
where $M$ is the Manning coefficient, generally taken to be $0.025$. 
Comparisons with and without friction have been performed in \Sec{bp9} for
Benchmark Problem \#9.

Coriolis terms can be turned on in GeoClaw, but have not been used for
any of the benchmark problems.

\subsection{Methods implemented in GeoClaw}
GeoClaw is a variant of the Clawpack open source
software \cite{claw.org} that LeVeque and collaborators have
been developing since 1994.  The ``wave-propagation algorithms'' used in
this software are described in great detail in the textbook
\cite{rjl:fvmhp} and in several journal publications
\cite{rjl:advect,rjl:waveprop,jol-rjl:3d}.


Adaptive mesh refinement (AMR) has been
incorporated since the inception of this software, 
in joint work with Marsha Berger at the
Courant Institute of Mathematical Sciences, one of the foremost authorities
on block structured AMR technology, often referred to as
Berger-Colella-Oliger style AMR \cite{berger-colella, berger-oliger}.  The
AMR algorithms in Clawpack are described in detail in
\cite{mjb-rjl:amrclaw}.  
Berger has also played a significant role in adapting the AMR routines to
work well in GeoClaw \cite{BergerGeorgeLeVequeMandli:awr11,
LeVequeGeorgeBerger:an11} in connection with well-balancing and inundation
modeling.

The GeoClaw software is described in some detail in
\cite{BergerGeorgeLeVequeMandli:awr11}, an invited contribution to a special
issue of
{\it Advances in Water Resources} on
``New Computational Methods and Software Tools'',
and in \cite{LeVequeGeorgeBerger:an11}, an invited paper in {\it Acta
Numerica}, which serves as an ``annual review of numerical analysis''.
There is also 
on-line documentation \cite{geoclaw-doc} that is part of the extensive
documentation of Clawpack.
The Riemann solvers and inundation model are desribed in more detail in
\cite{dgeorge:phd,George2008}

The open source Clawpack software (including GeoClaw) can be freely
downloaded from the website \url{http://www.clawpack.org}.  
Recent developments have taken
place using a Subversion repository that is openly accessible (linnked from
the Clawpack webpage), using issue tracking and a wiki, as well as the {\tt
claw-dev} google group to discuss development issues.  Clawpack is currently
transitioning to use of the Git, a more modern distributed version control
system, and the future home of Clawpack is on Github
(\url{https://github.com/organizations/clawpack}).

GeoClaw was initiated in the PhD work of David George
\cite{dgeorge:phd,rjl-george:catalina04a,dg-rjl:tsunami06,George2008} and
was originally called TsunamiClaw.  It has more recently been extended to
handle other geophysical flows such as storm surge \cite{mandli:phd}
dam break problems \cite{George:Malpasset}, and debris flows
\cite{GeorgeIverson2011}.

The GeoClaw model solves the two-dimensional nonlinear shallow water equations 
using high-resolution finite volume methods.  Values of $h$, $hu$, and $hv$
in each grid cell represent cell averages of the depth and momentum
components.  With flat bathymetry, the methods are exactly conservative for
both mass and momentum, and conserve mass for arbitrary bathymetry when used
on a fixed grid. They do not exactly conserve mass during regridding when
AMR is used because, for example, a grid cell that is dry on a coarse grid
may contain part of the shoreline on a finer grid.  This is not an issue as
long as the region of interest is refined before the main wave arrives.


These methods are based on Godunov's method, which means that at each cell
interface a one-dimensional Riemann problem is solved normal to the edge,
which reduces to a one-dimensional shallow water model with piecewise
constant initial data,
with left and right values given by the cell averages on each side.
The jump in bathymetry between the cells is incorporated into the Riemann
solution in a manner that makes the method ``well balanced'': the steady
state of the ocean at rest is exactly maintained.  This is done using the
``f-wave formulation'' of the wave propagation method, as discussed in
\cite{BaleLeVequeEtAl2002,George2008,rjl:fvmhp,rjl:wbfwave10}.


Godunov's method consists of solving the Riemann problem and using the
resulting wave structure to update cell averages in the adjacent finite
volume cells.  In practice an approximate Riemann solution is used, which
reduces to the standard Roe solver \cite{rjl:fvmhp,roe:rs}
in f-wave form in general, but is modified
to also handle dry states in order to model inundation \cite{George2008}.

Accuracy is improved by adding in high-resolution correction terms.  These
terms are based on Taylor series approximation of the exact solution at time
$t_{n+1}$ about time $t_n$, using the same approach as used to derive the
classical second-order accurate Lax-Wendroff  method \cite{rjl:fdm}.
However, the terms are only added where the solution is smooth.  Near steep
gradients, these terms lead to severe numerical dispersion and potentially
undershoots or overshoots that can lead to instability or unphysical states
(e.g. negative fluid depth).  A limiter is applied to the correction term
using the standard approach described in detail in 
\cite{rjl:fvmhp} and references found there, based on the theory of total
variation diminishing (TVD) methods that has been well developed since the
early 1980s.  The resulting ``high-resolution'' shock-capturing 
methods exhibit minimal numerical dispersion or dissipation and have been
shown to be very robust for both linear and nonlinear hyperbolic problems, 
even when the solution contains strong shock waves.

\subsection{Boundary conditions}\label{sec:bc}
Boundary conditions at edges of the computational domain are handled as
described in Chapter 7 of \cite{rjl:fvmhp}.
At the start of each time step, solution values are assigned in two
rows of ghost cells surrounding the computational domain.  This allows
the high-resolution finite volume methods to be applied on all cells
that lie inside the computational domain.  One row of ghost cells is
required in order to solver Riemann problems at edges of the physical
cells adjacent to the boundary.  The second row is required in order
to apply limiters.
Here we summarize the boundary conditions used in the benchmark
problems.  

\subsubsection{Non-reflecting boundaries}\label{sec:bc-outflow}
Zero-order extrapolation from the grid cells along the boundary to ghost
cells in every time step is used to implement non-reflecting boundary
conditions, for example when truncating the ocean or wave tank.  The
Godunov-type methods implemented in GeoClaw solve Riemann problems at each
grid interface and having equal values in the grid cell at the boundary and
the adjacent ghost cell results in no incoming wave.  These boundary
conditions are described in more detail in
Section 7.3.1 of \cite{rjl:fvmhp}.  Although not perfectly absorbing
for waves hitting the boundary at a non-normal angle, they perform
very well in practice and have been extensively used for similar
problems.

\subsubsection{Solid wall boundaries}\label{sec:bc-solid}
Several benchmark problems are posed in wave tanks with solid (reflecting)
walls.  Solid wall
boundary conditions are implemented as described in Section 7.3.3 of
\cite{rjl:fvmhp}.  The values in each grid cell adjacent to the boundary are
extrapolated into ghost cells and then the normal velocity is negated.  When
solving the Riemann problem, this anti-symmetric setup results in a Riemann
solution with zero normal 
velocity at the interface, modeling the correct boundary 
condition at this boundary.

\subsubsection{Inflow boundaries}\label{sec:bc-inflow}
Some benchmark problems specify an incoming wave, typically by tabulated
values of the depth at a gage near the inflow boundary.  
This can be implemented by filling ghost cells at each time step
with the desired values of the fluid depth and momentum values that are
determined using the Riemann invariants for the shallow water equations,
by assuming the depth and momentum are related in such a way that the
solution is a simple wave in the incoming wave family.  

For example, at
the left edge of the computational domain, an incoming wave would be a
right-going wave with constant values of the
Riemann invariant $u - 2\sqrt{gh}$ throughout the wave.
This value must be $-2\sqrt{gh_0}$
where $h_0$ is the depth of the undisturbed water before the wave arrives.
From this the velocity $u = 2(\sqrt{gh} - \sqrt{gh_0})$ can be determined
for any depth $h$.  Hence the given depth as a function of time at a
wave gauge can be used to determine the depth and momentum at this
point as a function of time, which in turn is used to fill ghost
cells.

\subsubsection{Initial conditions}\label{sec:ic}
In other problems, an initial wave form $h(x,t=0)$ is specified as a
function of $x$ at time $t=0$, representing a propagating wave form.
Again the Riemann invariants can be used to
determine the momentum at each point in order to specify initial
conditions.

For Benchmark Problem \#9 (\Sec{bp9}) the seafloor displacement is
specified.  This is what is generally done in real applications.  GeoClaw
allows specifying a time-varying seafloor displacement as well, to model the
dynamic rupture of faults.  This capability has been used in the landslide
problems (\Sec{bp3} and \Sec{bp8a}) as well to model the changing
bathymetry.



\subsection{Other validation studies}

Several of the benchmark problems were solved using TsunamiClaw, an early
version of GeoClaw, in preparation for the Catalina
benchmarking workshop in 2004.  These results are available in the
proceedings paper \cite{rjl-george:catalina04a}.

Validation studies of the more 
recent GeoClaw software have been presented in several
peer-reviewed papers:

\begin{itemize}
\item In \cite{BergerGeorgeLeVequeMandli:awr11} a test problem is used that
consists of a radially symmetric ocean with a continental shelf, and
a radially symmetric initial
hump of water at the center.  At one position along the coast a 
island is placed on the shelf.  Several gauges are located near the island.
If the island is rotated to a different position on the coast the results at
the gauges should be identical.  Numerically they will not be identical due
to differences in how the coast intersects the Cartesian grid.  Very similar
results are obtained for locations near an axis and near the diagonal, both
in gauge results and in plots of the surface and inundation.  This is true
even when AMR is used to concentrate fine grids only in the direction
towards the island.  In this paper the Chile 2010 tsunami is also used as a
test problem and good agreement with measured results at DART Buoy 32412
are obtained, on two different grid resolutions to test convergence.
See also 
\url{http://www.clawpack.org/links/awr11/}
for codes and animations.


\item Further tests of this same problems are presented in
\cite{LeVequeGeorgeBerger:an11}.  
See also 
\url{http://www.clawpack.org/links/an11/}
for codes and animations.

\item Grid refinement studies and comparison with field data for the widely
studied Malpasset dam failure are presented in \cite{George:Malpasset}.

\item In \cite{mjb-dac-ch-rjl:amrsphere09} tests are performed for a
tsunami-like wave propagating on the full sphere, using a novel mapped grid
that covers the sphere with a logically rectangular finite volume grid.
Tests are presented in which the bathymetry and the initial conditions are
axisymmetric so that the solution should remain so and can be compared to
fine grid one-dimensional simulations.  
See also 
\url{http://www.amath.washington.edu/~rjl/pubs/amrsphere09/index.html}
for codes used in this paper.
\end{itemize} 


Independent validation of GeoClaw has also been performed by Spatial Vision
Group (\url{http://www.spatialvisiongroup.com/}), a private consulting
company in Vancouver, BC. David Alexander and Bill Johnstone from this group
used the  TsunamiClaw software to perform hazard
studies of the communities of Ucluelet and Tofino on Vancouver Island.  As
part of their validation, they compared  TsunamiClaw results to those obtained
using MOST for a standard model of a Magnitude 9.0 
Cascadia Subduction Zone event.  Comparisons were performed for several tide
gauges on the coast of Washington and Oregon.

Our group is currently using GeoClaw to model the Great Tohoku Tsunami of 11
March 2011 and preliminary comparisons with DART buoy data looks very good.
Some results were posted online as computed in the days after the event and
can be viewed at
\url{http://www.clawpack.org/links/honshu2011/}.
Some comparisons were also presented in a recent article in SIAM News on tsunami
modeling \cite{siamnews-tsunami}.  Preliminary results have also been
obtained by other groups, e.g. \cite{ZhangYuenEtal:2011}.


\section{Benchmark results}
\ignore{(For each selected (PMEL) benchmark problem, please document how you defined
your model input data, grid and various other parameters to simulate each
problem. 

Provide results as figures (and raw data) following future formatting
instructions, to the relevant site and/or coordinator. Also summarize your
results as simple error indicators of the performance of the model (e.g.,
mean, max, rms errors).)}

The sections below contain the GeoClaw results for each benchmark problem.
Benchmark problem descriptions can be found in the Github repository
\url{https://github.com/rjleveque/nthmp-benchmark-problems}
\cite{bp-description} 
along with data that was provided as part of the problem specification.

